\part{Results}
\label{pa:results}
\chapter{Graph analysis}
PPI network from irefweb.
downloaded with these settings: %picture from irefweb
irefweb said 109276 interactions 
after hgnc it was 43706 (perturbation: 60\% from iref)
after clustering it was 13183 (perturbation: 87,9\% from iref - 69,8\% from hgnc) % not used
only uniprot and refseq
uniprot said
\section{Creating a network}
Converting from proteins to genes through HGNC data resulten in a 60\% data
network edge perturbation. From 100k ish links to 40k ish.
\subsection{Creating connections}
\subsection{Adding weights}
\section{Ranking results}
Talking about \gls{mcl} results
\begin{table}[H]
    \centering
    \begin{tabular}{| l | c | c | c | c | c | c |}
        \hline
        \textbf{Inflation} & \textbf{Clusters} & \textbf{Avg.  cluster size} &
        \textbf{Max. cluster size} & \textbf{Min. cluster size} &
        \textbf{Modularity} \\
        \hline
        1.6 & 1068 & 8.88 & 968 & 2 & 0.367 \\
        1.8 & 1400 & 6.60 & 660 & 2 & 0.307 \\
        2.0 & 1599 & 5.68 & 405 & 2 & 0.269 \\
        2.5 & 2053 & 4.20 & 179 & 2 & 0.223 \\
        3.0 & 2210 & 3.75 & 122 & 2 & 0.199 \\
        \hline
    \end{tabular}
    \caption{MCL clustering parameter and statistic results}
    \label{tab:mcl-inflation}
\end{table}
The modularity of the clustered networks gives an indicator of how well the
process of creating the clusters went. Modularity is given as a score from 0 to
1. A score closer to 1 is more preferrable, as this indicates that the clusters
created have a good degree of separation to the other clusters in the network.
The preferred score to end up with would be around 0.8, but in this network
there has been a good amount of perturbation through the protein-to-gene
process. Modularity is not the only indicator of how well a network was
clustered, hence the choice of not setting the inflation value in \gls{mcl} to
1.6, but rather 1.8. When a lower inflation value is set, \gls{mcl} does not
separate edges between nodes as vigorously and as a direct cause, inflation will
go up. Taking the other attributes in the table \ref{tab:mcl-inflation} into
consideration, 1.8 seemed like the best inflation value. An inflation value of
1.8 has also been proved to be good for large high-throughput constructed
protein-protein networks with a large amount of alterations\cite{mcl-inflation}.
%% Move this part to methods?

\section{Cross-validation}
Executing a cross-validation on the iRefWeb with PRWP and MAA rankings had two
purposes. The first being to prove the fact that every gene that had its prior
score removed by the cross-validation, should be found in the results of the
cluster ranking and identified as candidate biomarkers. 

The second purpose was to analyze the distribution of the genes with prior
scores removed by the cross-validation in terms of how they placed in the
cluster ranking. The analysis of this distribution was done by dividing the
amount of genes removed by cross-validation in a cluster by the total amount of
genes in the same cluster. This operation was repeated for every cluster in the
ranking, resulting in a distribution of the average amount of genes removed by
cross-validation, that was detected by Ranklust together with post-processing of
data, as cancer candidate biomarkers. A distinct descending distribution from
high to low ranked clusters would indicate that most of the genes, with a prior
score of their relevance to prostate cancer, was ranked in a way that achieved
the goal of Ranklust; ranking clusters in biological network according to
network structure and prior knowledge of relevance to diseases.

\subsection{Cross-validation in PRWP}
\hspace*{-1cm}\begin{figure}[H]
    \label{fig:irefweb-prwp}
    \includegraphics[scale=0.6]{cv_dist_total_filtered_prwp}
    \caption{Cross-validation distribution in clusters (PRWP)}
\end{figure}
This plot \ref{fig:irefweb-prwp} is developed from the 10 random
cross-validation runs ranked with PRWP. The results from this cross-validation
shows that the higher the rank of the cluster, the more candidate cancer genes
the cluster had.

\subsection{Cross-validation in MAA}
\hspace*{-1cm}\begin{figure}[H]
    \label{fig:irefweb-maa}
    \includegraphics[scale=0.6]{cv_dist_total_filtered_maa}
    \caption{Cross-validation distribution in clusters (MAA)}
\end{figure}
This plot \ref{fig:irefweb-maa} is developed from the 10 random
cross-validatioon runs ranked with MAA. The results from this cross-validation
shows that the higher the rank of the cluster, the more candidate cancer genes
the cluster had.

\subsection{PRWP versus MAA}
There is clearly a trend in both \gls{prwp} and \gls{maa}. The difference
between them being mainly the amount of clusters found and the coupling of
values around the linear regression fit. It is important to point out that this
is just a result over the distribution of candidate cancer genes at certain
ranks. Where the clusters resides in the rankings may be very different between
the \gls{prwp} and \gls{maa}.

The fact that both of the ranking algorithms managed to get a descending amount
of prostate candidate cancer genes in the clusters, which was cross-validated
from the same \gls{golden} priors, indicates that a certain similarity of what
was mentioned in the previous paragraph. The similarity implicated here is how
different the ranking of the clusters are between \gls{prwp} and \gls{maa}.

The values in \gls{prwp} have a tighter coupling, in other words, the distance
the coordinates have in the scatter plot deviate less from the fit than the ones
in the \gls{maa} plot. The difference is not huge, as the coefficient of
determination indicates, 0.336 in \gls{prwp} against 0.332 in \gls{maa}. Both
ranking algorithms achieves a descending distribution of cross-validated genes
from the topmost to the lowest ranked cluster. But when comparing the
cross-validation results between \gls{prwp} and \gls{maa}, \gls{prwp} comes out
ahead by a margin in \gls{rsquared} value.

\section{Benchmarks}
In all of the upcoming plots of benchmarks, blue will represent prostate cancer
biomarkers from the \gls{golden} and orange represent prostate candidate cancer
biomarkers. The benchmarks are all based on prostate cancer data queried from
the \gls{jensen} database\cite{jensen}. The database has a hierarchy based
definition of diseases, so filtering the data on an insensitive-case "prostate
cancer"-query with the UNIX grep tool\cite{grep} retrieved all of the genes used
for the benchmark.

The average z-values in the clusters are split among 2 categories as mentioned
above. The z-values of the prostate candidate cancer genes in a cluster is
calculated from the average of all of them. The same procedure was done for the
z-values of the prostate cancer genes.

\subsection{Text mined and scored with z-values}
These next z-value scored plots represents text mined gene results from the
\gls{jensen} database.

\begin{figure}[H]
    \label{fig:txt-iref-prwp}
    \includegraphics[width=15cm]{prwp_textmined_split}
    \caption{Average distribution of z-scores in clusters ranked by PRWP.}
\end{figure}
For \gls{prwp}, the prostate cancer candidate genes is descending in z-values
from the topmost ranked cluster to the lowest, which is contributing to
showing \gls{prwp}s suitability for ranking clusters as candidate biomarkers. 

The prostate candidate cancer biomarkers are ascending in z-value from the
topmost to the lowest ranked cluster. High z-values would contribute to the fact
that ranklust has found actual prostate candidate cancer biomarkers. However,
a low z-values does not contradict it. The only fact to deduce from low z-values
is that they have not been examinated to the degree that they are not mentioned
in papers related to prostate cancer.

\begin{figure}[H]
    \label{fig:txt-iref-maa}
    \includegraphics[width=15cm]{maa_textmined_split}
    \caption{Average distribution of z-scores in clusters ranked by MAA.}
\end{figure}
For \gls{maa}, the prostate cancer genes have the same distinct descension in
z-values from the topmost to the lowest cluster ranks. The difference from
\gls{prwp} to \gls{maa} being that \gls{maa} has a more distinct ascending
linear regresstion fit for the z-values in the prostate candidate cancer genes.

This demonstrates the main difference between \gls{prwp} and \gls{maa}.
\gls{prwp} takes network structure into comparison as well as the prior scores.
\gls{maa} focuses purely on the prior scores, and so the network structure of
protein complexes are being completely ignored.

\subsection{Knowledge curated distribution of genes}
This data had no score except for a confidence score in the \gls{jensen}
database. Due to the low grade of distinctiveness between the different genes in
the database, the average amount of genes in a cluster would receive a knowledge
score based on if the gene occurs in the knowledge curated part of the
\gls{jensen} database or not. So the knowledge curated data is only based on
occurence, and not a specific value, in contrast to the text mined and
experimentally mined genes in the database.

Another trait the knowledge curated data posssesses is the amount entries in the
\gls{jensen} database that has. Text mined data can be seen as the
high-throughput technology of retrieving relevant data from papers, while the
knowledge curated data is manually curated by researchers. This is why the
amount of entries for knowledge curated data is so low when compared to text
mined data.

\begin{figure}[H]
    \label{fig:know-iref-prwp}
    \includegraphics[width=15cm]{prwp_knowledge_split}
    \caption{Average distribution of curated knowledge mined genes in clusters
    ranked by PRWP.}
\end{figure}
-- PRWP knowledge curated data discussion --

\begin{figure}[H]
    \label{fig:know-iref-maa}
    \includegraphics[width=15cm]{maa_knowledge_split}
    \caption{Average distribution of curated knowledge mined genes in clusters
    ranked by MAA.}
\end{figure}
-- MAA knowledge curated data discussion --

\subsection{Experimental mined genes distribution of p-values in genes}
\begin{figure}[H]
    \label{fig:exp-iref-prwp}
    \includegraphics[width=15cm]{prwp_experimental_split}
    \caption{Average distribution of p-values in clusters ranked by PRWP.}
\end{figure}
-- PRWP experimentally mined data discussion --

\begin{figure}[H]
    \label{fig:exp-iref-maa}
    \includegraphics[width=15cm]{maa_experimental_split}
    \caption{Average distribution of p-values in clusters ranked by MAA.}
\end{figure}
-- MAA experimentally mined data discussion --


\section{Comparison to known biomarkers}
-- Tests from Movember data etc. --

\section{Identification of possible cluster biomarkers}
-- Final list with cluster biomarkers --
