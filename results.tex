\part{Results}
\label{pa:results}
\chapter{Neo4J}
Neo4J was chosen as the database I used to interact with the data from
\textit{CytoScape}.

% Neo4J refactor script
I ended up with implementing a python script to refactor the Neo4J database
dump. When run with the command:
\begin{verbatim}
time python relations_refactoring.py movember.cql refactored_movember.cql
\end{verbatim}
The inputs here are time, which benchmarks the time and CPU power used by the
program, python which runs the script, the name of the script, inputfile that we
want to refactor and the name of the file, which the refactored output resides
in.  The runtime resulted in:
\begin{verbatim}
python relations_refactoring.py movember.cql refactored_movember.cql
0,04s user 0,00s system 42% cpu 0,092 total
\end{verbatim}
The size of the refactored file was 4500 lines, where about half of them were
queries that created relationships between two nodes. The size of the database
that was dumped was 2,11 megabytes according to the Neo4J browser interface.
Also, the program used 500 megabytes of memory when run on my personal computer
with Python 3.4. The queries that created the nodes was bundled into several
commits, which ended up being approximately 500 queries per commit. The
relationship queries was singled out and can be seen as a single commit for each
relationship created between two nodes. This can be improved further, but at the
time it was not necessary, as a runtime below 1 second does not introduce any
difficulties or problems at the time.

\section{Data communication}
Data communication between the Neo4J database server and the CytoScape instance
is done through the CyNeo4J app to CytoScape \cite{cyneo4j}. CyNeo4J is a
CytoScape app that was developed during the Google Summer Code 2014 arrangement.
It supports connecting to a Neo4J instance, as well as syncing data up and down
from and to the database server. One thing it did not support was authentication
on Neo4J servers. Not having authentication is a serious problem, so I
implemented a simple way of getting access to the database server by providing a
possibility to insert username and password at the same time the user has to
provide a URL to the Neo4J database server instance. Implementation-wise, this
only required an extra header to be included in each http request going to the
password protected Neo4J database server instance. Every request used the static
\textbf{Request} class to Get/Post/Put HTTP requests to the Neo4J database
server instance. The refactor looked something along these lines in all of the
files.

Before:
\begin{minted}{Java}
Request req = Request.Post(url)
        .bodyString(call.getPayload(), ContentType.APPLICATION_JSON);
\end{minted}

After:
\begin{minted}{Java}
Request req = Request.Post(url)
        .addHeader("Authorization", auth64EncodedInfo)
        .bodyString(call.getPayload(), ContentType.APPLICATION_JSON);
\end{minted}

Synchronization time between Neo4J and CytoScape through the CyNeo4J app is a
huge timesink. As of now, the time it takes to populate an empty CytoScape
network with the gene information from STRING is about 2 hours. This could be
shortened by exporting the Neo4J data with GraphML and into CytoScape. Because
after the initial data is inside CytoScape, updates to the Neo4J instance goes
much faster.

The CyNeo4J also uses a legacy HTTP library to get information from the Neo4J
database \cite{legacy-neo4j}. It is possible that the performance increases with
the new library \cite{transactional-neo4j}. The new library supports creating
transactions, which implicit gives support for rollbacks in case something goes
wrong with the query.

A future improvement to the CyNeo4J app could be to change the communication
between Neo4J and CytoScape to be done in GraphML and not Cypher. This is
because through this whole process of importing and exporting data from and to
Neo4J, GraphML has shown itself to be a superior format over Cypher. Though, to
this day, direct queries to a running Neo4J instance has to be done in Cypher,
and is not possible in GraphML. So until the Cypher query language gets faster,
a convertion method using GraphML as way of exchange data between CytoScape and
Neo4J, could be a candidate for a much better solution than the current one,
which is 100\% Cypher query based.

\chapter{Graph analysis}
PPI network from irefweb.
downloaded with these settings: %picture from irefweb
irefweb said 27887 connections % Correct this!
only uniprot and refseq
uniprot said
\section{Creating a network}
\subsection{Creating connections}
\subsection{Adding weights}
\section{Parameter testing}
\section{Comparison to known biomarkers}
\section{Identification of possible biomarkers}
