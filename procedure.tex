\part{Procedure}
The secretome values of genes can be expressed by a single integer value,
though, for compatibility reasons Ranklust will require double values. The
clustered network can be constructed with an AP, short for Affinity Propagation,
clustering algorithm \cite{affinity-propagation}. Affinity propagation
clustering concentrates on the nodes in the network that are binding the rest of
them together. Using affinity propagation for clustering will produce results
that represent a grouping of nodes that are coupled by seemingly unimportant
nodes to most clustering algorithms. But the AP algorithm is good at expressing
nodes that are not highly connected to many nodes, but rather the nodes that are
binding other highly connected nodes together. This results in bigger clusters
that can be a target for methods that cure cancer by severing the interaction
between biomarker genes.

It is also discussed how AP performs versus Markov clustering (source, both
markov algorithm and the "vs" paper). And since Markov algorithm performs better
on protein interaction, it will also be used to cluster the networks. An
analysis between the rankings that come from the Markov and AP clusterings will
be performed, which hopefully will give concrete results as to the pro's and
con's of each algorithm. Some questions should be raised as the analysis is
done. For example, is both of the algorithms good, but have different uses, even
if they are not directly involved with the ranking done after the clustering?
Are either AP or Markov useless for a particular type of ranking afterwards?

The information provided through the whole process from what type of nodes
(protein or gene), interaction between the nodes and what kind of biomarker we
want to identify are all factors that will have a great impact on how all of
this should be combined. The order of operations on the network will also effect
the result. For example, AP clustering may create a few big clusters and many
small ones. At this stage, the results will consist of the biggest clusters
constructed from AP clustering that has focus on pure connection between nodes
and not their attributes. At this point, the cluster ranking algorithm of
Ranklust will be run in order to produce a picture of potential biomarkers. This
picture is the first and simplest step that will be used as a result for an
analysis. The analysis in this thesis will focus on validating the cluster
ranking scores as ways of indicating potential biomarkers. 

In order to validate the scores, I will need to know the state of the patient
that the data to create the network came from. As there is no use to just
generate results without knowing what they show. They might show connections
between them, but without some sort of context the results are useless. The
context needed is not very high, but the results from the ranking should be
tested for several purposes. For example, is biomarkers from ranking of clusters
best for disease disposition, screening, diagnostic, prognostic, prediction or
monitoring cancer. I will aim for screening, diagnostic and most of all
prognostic usages. As mentioned earlier, the prognostics of prostate cancer
often results in 50\% of patients receiving treatment that was not needed.

