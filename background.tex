\part{Background}
\section*{Introduction}
We have our body, and inside our body we have our organs. These organs are made
up of tissue, and tissue is made up of cells. Our cells performs two type of
functions, to execute chemical reactions needed to stay alive, and to pass
information for maintaining life onto the next generation.
\section*{Background}
\subsection*{DNA, RNA and protein in general}
The name of the process when DNA goes from DNA to RNA to protein is called the
\textbf{Central Dogma}. I will focus on the cell, how we perceive it and the
interaction inside it. The cell can be seen as a network community of
interacting protein molecules. The protein comes from our DNA.

In our DNA, there are areas that contain codes for making protein. These areas
are called genes. Also, all of the possible interactions between the cells are
specified by the proteins in complex social networks. The reason for calling
these networks for "social networks" is because it is the interactions that we
look at as the connections between the cells. It is our protein that performs
chemical reactions in our body. DNA on the other hand stores and passes
information about how our body is built up. RNA is the intermediate stage
between DNA and protein.

\subsection*{Protein}
The protein in our body is made up from amino acids. The amount of amino acids
in a protein may vary from 20 to 5000.  But on average there is about 350 amino
acids in a protein in our body. The way we identify amino acids is through the
alphabet. We use almost every letter in the alphabet, and it is organized in a
chronological order when we show them in a list, but we miss some characters.
They can also be identified by three letters, or the whole name of the amino
acid.

Amino acids have some very basic attributes like volume and mass. But beeing
acids we also have information about their polarity and their basicity/acidity.
So the amino acids are either polar or non-polar, combined with beeing neutral,
acidic or basic. And of course we are able to see to what degree they are
acidic/basic.

Amino acids consist of an amino group, carboxyl group and a R group. Very often
there is also a central carbon that combines all of these groups together. We
have about 20 different amino acids and they can be classified into 4 types.
The positively charged, the negatively, the polar and the non-polar. The
positive amino acids are basic and the negative ones are acidic. The four types
just mentioned are not totally exclusively to eachother, though an amino acid
cannot be both basic and acidic at the same time. At the same time an amino acid
cannot be both polar and non-polar. But any combination of acidity/basicity
together with polarity can occur.
\subsection*{Genome}
When the genome changes suddenly and unexpected, we have a mutation. A mutation
may happen when several things occur:
\begin{itemize}
    \item Insertion
        \begin{itemize}
            \item A part of a chromosome gets inserted into another one
        \end{itemize}
    \item Deletion
        \begin{itemize}
            \item A part of a chromosome gets deleted
        \end{itemize}
    \item Duplication
        \begin{itemize}
            \item A part of a chromosome gets duplicated
        \end{itemize}
    \item Inversion
        \begin{itemize}
            \item A part of a chromosome gets inverted, but yet it stays in
                place
        \end{itemize}
    \item Translocation
        \begin{itemize}
            \item Two chromosomes exchanges parts and they become what is called
                a derivative
        \end{itemize}
\end{itemize}
