\part{Implementation}
\label{pa:implementation}
\chapter{CytoScape}
The pull request\cite{git-pull-request} made from the Ranklust
contribution of this thesis has been accepted\cite{ranklust-accepted}, adding
3750 lines of code and deleting 1051. The changes was distributed across 50
files. The type of files edited ranges from the Maven POM-settings file and
.gitignore to plain Java source code files. Further changes will be made, as the
work on the Ranklust contribution will not end when this thesis is delivered.
One of the most recent changes to the contribution is that the
\textit{PageRankWithPriors} algorithm used to analyze the networks in this
thesis assumes undirected edges because of the
\textit{UndirectedSparseMultigraph} graph used, while for normal use, directed
edges is believed by us to be the most sought after choice. Therefore, the pull
request was hardcoded with directed edges. Choosing between directed and
undirected edges should be up to the users of clusterMaker2 and is on the
TODO-list of futures to implement.

CytoScape has a cookbook\cite{cytoscape-cookbook} listing a combination of best
practices and tips on how to start developing an app, add menues, panels,
algorithms, color schemes etc.. Instead of following all of these examples, we
have read through the existing code in clusterMaker2\cite{cm2-github} and tried
to structure the code of the contributions in Ranklust to match the structure
already implementet in clusterMaker2. Meaning that for each algorithm
implemented, they will all have a corresponding \textit{Factory} class and a
\textit{Context class}. For the panel that was implemented, it has a pure panel
class with Java Swing\cite{java-swing} settings, a panel task class, create
panel task class and destroy panel task class. % Make UML!

\section{App registration and java class connections}
First step of registration is to register a \textit{service listener} to the
already existing clusterMaker2 \textit{CyActivator}. This is done in the last
line of the coming code example: Here we describe the name of java methods in
the \textit{clusterManager} class with strings. These two methods for adding and
removing ranking algorithms registers and unregisters the algorithms to the
\textit{App} menu in CytoScape, making them accessible for the user through the
GUI.

\paragraph{Registering the ranking algorithms classes}
\inputminted[linenos,fontsize=\scriptsize,firstline=65,lastline=93]{java}{../ranklust/src/main/java/edu/ucsf/rbvi/clusterMaker2/internal/CyActivator.java}

\paragraph{Registering the ranking panel create and destroy classes}
\inputminted[linenos,fontsize=\scriptsize,firstline=173,lastline=180]{java}{../ranklust/src/main/java/edu/ucsf/rbvi/clusterMaker2/internal/CyActivator.java}

The \textit{RankFactory} class mentioned is a java public interface for each of
the ranking cluster algorithms, HITS, PR, PRWP, MAA and MAM. 

% related to paragraph over
\inputminted[linenos,fontsize=\scriptsize,firstline=3,lastline=10]{java}{../ranklust/src/main/java/edu/ucsf/rbvi/clusterMaker2/internal/api/RankFactory.java}

A class applying the task factory design pattern is meant to deliver an object
of the class it is related to\cite{factory-design}, and it can return different
types of a class that has the same java superclass\cite{java-superclass}. In
this case, each class implementing the \textit{RankFactory} will only have a
single class to return. The RankFactory class also creates a \textit{Context}
class, which binds information the user can input to the GUI to variables,
allowing the algorithm to gain access to parameter information about the
algorithms before they run.

% Register algorithms task factories
\inputminted[linenos,fontsize=\scriptsize,firstline=144,lastline=149]{java}{../ranklust/src/main/java/edu/ucsf/rbvi/clusterMaker2/internal/CyActivator.java}

With these factories, the underlying algorithm returned can be changed based on
parameters

% Register algorithms in GUI
\inputminted[linenos,fontsize=\scriptsize,firstline=186,lastline=207]{java}{../ranklust/src/main/java/edu/ucsf/rbvi/clusterMaker2/internal/ClusterManagerImpl.java}

% Register GUI panels in GUI
\inputminted[linenos,fontsize=\scriptsize,firstline=260,lastline=282]{java}{../ranklust/src/main/java/edu/ucsf/rbvi/clusterMaker2/internal/ClusterManagerImpl.java}
\section{Algorithm}

% Rank interface
\inputminted[linenos,fontsize=\scriptsize,firstline=260,lastline=282]{java}{../ranklust/src/main/java/edu/ucsf/rbvi/clusterMaker2/internal/api/Rank.java}
\section{GUI}
\subsection{Algorithm menus}
Much of these menus
\subsection{Ranking panel}
The code for the \textit{RankingPanel} is a copy from the existing
\textit{ResultsPanel} in clusterMaker2. The results panel has some extra
information that is removed in the ranking panel. The ranking panel only
displays which clustering and ranking algorithm that was used and the network it
is used on, together with the score of each cluster sorted descendingly, having
the highest ranked cluster at the top. The ranking panel will display in the
same place as the results panel, to the right of the network view.

The panel supports multiple selection of clusters through the \textit{Control}
key on the keyboard, much like its existing behaviour on Windows when selecting
multiple directories or files in the \textit{File Explorer}. Selecting the
clusters in the panel will also select them in the network view, enabling the
user to use other CytoScape utilities on the nodes/clusters selected, for
example create a new network based on only the selected nodes. The color of the
selected nodes will also change in the network view, when the user selects a
cluster in the ranking panel, providing visual feedback to the user, in order to
make it easier for the user to see where the selected nodes/clusters resides in
the network.
\inputminted[linenos,fontsize=\scriptsize,firstline=260,lastline=282]{java}{../ranklust/src/main/java/edu/ucsf/rbvi/clusterMaker2/internal/ui/RankingPanel.java}

When tasking clusterMaker2 with showing the ranking panel, the color of the
nodes will also change. This is to visualize the rank of each cluster in the
network view to the user. Coloring the highest scoring clusters with red or
green and the lowest scoring with the opposite was the first and easiest
solution. There was made a choice to color the whole cluster according to the
rank it received, instead of individually color nodes according the their
contribution to the clusters rank. The user might be interested in choosing what
suits them best when it comes to this, so it might be implemented a menu for the
ranking panel in the future to best meet the users needs.

Taking color blind people into consideration red and green is not the
best combination. Colors resembling red and green in the form of hot-to-cold
colors that all types of colorblind people can view was therefore a criteria to
be met. A style satisfying these criteria will follow, having the hex value of
te color\cite{color-blindness3}, followed by the RGB value that was calculated
using a site that converts from hex to RGB values\cite{color-blindness2}.
\textit{\#ADD rgb\(170,221,221\)} for a blueish and cold color representing low
score. \textit{\#FEC rgb\(255,248,204\)} represents off-brown color that is a step
warmer than the blueish. The warmest color representing the highest scored
clusters is \textit{\#F99 rgb\(255,153,153\)}. The drawback with this color
combination is with people that has color blindness to the degree where they see
only greyscales. The colors low to high will go from grey to light grey to dark
gray, which does not seem logical. The color style might change if users
experience trouble having these colors represent the cluster ranks. 

\section{Known bugs}
\subsection{Menu bugs}
\paragraph{Unintentional execution of clustering algorithm}
\begin{enumerate}
    \item Open the clusterMaker2 clustering algorithm menu
    \item Select a clustering algorithm
    \item Exit the menu for the algorithm without running it
    \item Repeat step 1-2 and it results in running the previous algorithm that
        was exited
\end{enumerate}
This was the behaviour before Ranklust contributions was added to clusterMaker2,
so it might be clusterMaker2 or it might be CytoScape itself. Most likely, the
way clusterMaker2 algorithms is started through its \textit{TaskFactory}'s needs
to be changed, but this is just a hunch and not based on any analysis.

\subsection{Ranking panel bugs}
It is a known bug with the style for node coloring that the ranking panel
causes. The colors in the nodes might flicker when several ranking algorithms
has been run and the colors of each node has changed several times. This is
probably related to the previous color style for the nodes and it will be fixed
in the future.
