\part{Implementation}
\label{pa:implementation}
\chapter{CytoScape}
The pull request\cite{git-pull-request} made from the Ranklust
contribution of this thesis has been accepted\cite{ranklust-accepted}, adding
3750 lines of code and deleting 1051. The changes was distributed across 50
files. The type of files edited ranges from the Maven POM-settings file and
.gitignore to plain Java source code files. Further changes will be made, as the
work on the Ranklust contribution will not end when this thesis is delivered.
One of the most recent changes to the contribution is that the
\textit{PageRankWithPriors} algorithm used to analyze the networks in this
thesis assumes undirected edges because of the
\textit{UndirectedSparseMultigraph} graph used, while for normal use directed
edges is believed by us to be the most sought after choice. Therefore, the pull
request was hardcoded with directed edges. Choosing between directed and
undirected edges should be up to the users of clusterMaker2 and is on the
TODO-list of futures to implement.

CytoScape has a cookbook\cite{cytoscape-cookbook} listing a combination of best
practices and tips on how to start developing an app, add menues, panels,
algorithms, color schemes etc.. Instead of following all of these examples, we
have read through the existing code in clusterMaker2\cite{cm2-github} and tried
to structure the code of the contributions in Ranklust to match the structure
already implementet in clusterMaker2. Meaning that for each algorithm
implemented, they will all have a corresponding \textit{Factory} class and a
\textit{Context class}. For the panel that was implemented, it has a pure panel
class with Java Swing\cite{java-swing} settings, a panel task class, create
panel task class and destroy panel task class.

\section{App registration and java class connections}
First step of registration is to register a \textit{service listener} to the
already existing clusterMaker2 \textit{CyActivator}. This is done in the last
line of the coming code example: Here we describe the name of java methods in
the \textit{clusterManager} class with strings. These two methods for adding and
removing ranking algorithms registers and unregisters the algorithms to the
\textit{App} menu in CytoScape, making them accessible for the user through the
GUI.

\paragraph{Registering the ranking algorithms classes}
\inputminted[linenos,fontsize=\scriptsize,firstline=65,lastline=93]{java}{../ranklust-app/src/main/java/edu/ucsf/rbvi/clusterMaker2/internal/CyActivator.java}

\paragraph{Registering the ranking panel create and destroy classes}
\inputminted[linenos,fontsize=\scriptsize,firstline=173,lastline=180]{java}{../ranklust-app/src/main/java/edu/ucsf/rbvi/clusterMaker2/internal/CyActivator.java}

The \textit{RankFactory} class mentioned is a java public interface for each of
the ranking cluster algorithms, HITS, PR, PRWP, MAA and MAM. 

% related to paragraph over
\inputminted[linenos,fontsize=\scriptsize,firstline=3,lastline=10]{java}{../ranklust-app/src/main/java/edu/ucsf/rbvi/clusterMaker2/internal/api/RankFactory.java}

A class applying the task factory design pattern is meant to deliver an object
of the class it is related to\cite{factory-design}, and it can return different
types of a class that has the same java superclass\cite{java-superclass}. In
this case, each class implementing the \textit{RankFactory} will only have a
single class to return. The RankFactory class also creates a \textit{Context}
class, which binds information the user can input to the GUI to variables,
allowing the algorithm to gain access to parameter information about the
algorithms before they run.

% Register algorithms task factories
\inputminted[linenos,fontsize=\scriptsize,firstline=144,lastline=149]{java}{../ranklust-app/src/main/java/edu/ucsf/rbvi/clusterMaker2/internal/CyActivator.java}

With these factories, the underlying algorithm returned can be changed based on
parameters

% Register algorithms in GUI
\inputminted[linenos,fontsize=\scriptsize,firstline=186,lastline=207]{java}{../ranklust-app/src/main/java/edu/ucsf/rbvi/clusterMaker2/internal/ClusterManagerImpl.java}

% Register GUI panels in GUI
\inputminted[linenos,fontsize=\scriptsize,firstline=260,lastline=282]{java}{../ranklust-app/src/main/java/edu/ucsf/rbvi/clusterMaker2/internal/ClusterManagerImpl.java}
\section{Algorithm}

% Rank interface
\inputminted[linenos,fontsize=\scriptsize,firstline=260,lastline=282]{java}{../ranklust-app/src/main/java/edu/ucsf/rbvi/clusterMaker2/internal/api/Rank.java}
\section{GUI}
\subsection{Algorithm menus}
\subsection{Ranking panel}
The code for the \textit{RankingPanel} is a copy from the existing
\textit{ResultsPanel} in clusterMaker2. The results panel has some extra
information that is removed in the ranking panel. The ranking panel only
displays the clustering algorithm used, ranking algorithm used and the network
it is used on, together with the score of each cluster sorted descendingly,
having the highest ranked cluster at the top. The ranking panel will display in
the same place as the results panel, to the right of the network view.

The panel supports multiple selection of clusters through the \textit{Control}
key on the keyboard, much like its existing behaviour on Windows when selecting
multiple directories or files in the \textit{File Explorer}. Selecting the
clusters in the panel will also select them in the network view, enabling the
user to use other CytoScape utilities on the nodes/clusters selected, for
example create a new network based on only the selected nodes. The color of the
selected nodes will also change in the network view, when the user selects a
cluster in the ranking panel, providing visual feedback to the user, in order to
make it easier for the user to see where the selected nodes/clusters resides in
the network.
\inputminted[linenos,fontsize=\scriptsize,firstline=260,lastline=282]{java}{../ranklust-app/src/main/java/edu/ucsf/rbvi/clusterMaker2/internal/ui/RankingPanel.java}
