\documentclass[a4paper,UKenglish]{ifimaster}
\usepackage[utf8]{inputenc}
\usepackage[UKenglish]{duomasterforside}
\usepackage[T1]{fontenc, url}
\usepackage{graphicx,babel,csquotes,textcomp,varioref,float}
\usepackage[backend=biber,style=numeric-comp,sorting=nyt,maxbibnames=20,maxcitenames=20]{biblatex}
\usepackage[toc]{glossaries}
\usepackage{listings}
\usepackage{xcolor}
\usepackage{parskip}
\usepackage{fancyvrb}
\usepackage{tabularx}
\usepackage{rotating}
\usepackage{multirow}
\urlstyle{sf}
\graphicspath{{images/}}
\addbibresource{macros.bib}
\addbibresource{references.bib}

\lstset{
    breaklines=true,
    frame=tb,
    numbers=left,
    stepnumber=1,
    numberstyle=\color{gray},
    keywordstyle=\color{blue},
    stringstyle=\color{red},
    commentstyle=\color{olive},
    keepspaces=true
}

\makeglossaries
\newglossaryentry{golden}
{
    name=golden standard,
    description={The prior scoring standard evolved from the DisGeNET and Dragon
    Database of Genes associated with Prostate Cancer}
}
\newglossaryentry{maa}
{
    name=MAA,
    description={The Multiple Attribute Additive method described to score
    clusters}
}
\newglossaryentry{mam}
{
    name=MAM,
    description={The Multiple Attribute Multiplicative method described to score
    clusters}
}
\newglossaryentry{pr}
{
    name=PR,
    description={The PageRank method described to score clusters}
}
\newglossaryentry{prwp}
{
    name=PRWP,
    description={The PageRank With Priors method described to score clusters}
}
\newglossaryentry{hits}
{
    name=HITS,
    description={The Hyperlink-Induced Topic Search method described to score
    clusters}
}
\newglossaryentry{maven}
{
    name=Maven,
    description={The Java build tool used to compile and build clusterMaker2 and
    Ranklust}
}
\newglossaryentry{ide}
{
    name=Integrated Development Environment,
    description={A development environment specialized in developing software
    products}
}
\newglossaryentry{mcl}
{
    name=MCL,
    description={Markov Cluster algorithm used in Cytoscape to cluster the
    networks}
}
\newglossaryentry{rsquared}
{
    name=R-squared,
    description={The coefficient of determination when talking about the grade
    of fitness of a linear regression}
}
\newglossaryentry{jensen}
{
    name=DISEASE,
    description={DISEASE database used for retrieving z-values, p-values and
    manually curated disease-gene associations}
}
\newglossaryentry{pipeline}
{
    name=pipeline,
    description={The time used from start to end when analyzing the network in
    Cytoscape}
}
\newglossaryentry{dragon}
{
    name=DDPC,
    description={Dragon Database of Genes associated with Prostate Cancer}
}
\newglossaryentry{movember}
{
    name=potential prostate cancer genes,
    description={Data retrieved from the Movember Prostate Cancer Project}
}

\title{Ranklust}
\subtitle{An extension of the Cytoscape clusterMaker2 plugin and its application
to prioritize network biomarkers in prostate cancer}
\author{Henning Lund-Hanssen}

\begin{document}
\duoforside[program={Programming and Networks},
    dept={Department of Informatics},long]

\frontmatter{}

\setlength{\parskip}{12pt}
\setlength{\parindent}{12pt}

\chapter*{Abstract}
Prostate-specific antigen (PSA) is a prostate cancer biomarker used for
screening that in a great number of cases has led to overtreatment. The main
flaw with this biomarker is that mutation or abnormality of a single gene is
seldom the single cause of a disease. Rather, it is the process of interactions
between several components in a complex network. Next-generation technologies
provide bioinformaticians with data on a scale worthy of being labeled "Big
Data". One of the advantages gained is the potential to identify and prioritize
genes that promote diseases by analyzing biologically related networks. Another
advantage is the newly gained understanding of the topological organization of
large-scale molecular networks, which has been achieved through clustering
networks, creating subnetworks representing functions. The goal of this
assignment was to combine these two newly adapted methods in bioinformatics to
develop a plugin for Cytoscape, a network data integration, analysis and
visualization tool, in order to prioritize network biomarkers for prostate
cancer. Through cross-validation, benchmarking against text mined, manually
knowledge curated and experimental data, Ranklust has demonstrated this ability
by using the PageRank with Priors algorithm to rank clusters made with the
Markov Cluster algorithm in a protein-protein interaction network (PPI). It
ended up being merged to one of the future branches of clusterMaker2 on GitHub.

\chapter*{Acknowledgements}
I would first like to thank my two thesis supervisors. Research Fellow and
supervisor Trevor Clancy of the Institute for Cancer Research at Oslo University
Hospital and Group leader and co-supervisor Eivind Hovig of the Institute of
Cancer Research at the Oslo University Hospital and Professor II of the
Institute of Informatics at the University of Oslo. You guys have always been
available when I needed advice, supported me throughout the whole process and
provided me with invaluable knowledge.

Finally, I must express my very profound gratitude to my family, for their
unfailing support and continuous encouragement. To my fellow study companions at
room 3417 Assembler "Assa", Ole-Johan Dahls hus, over the years you guys have
taught me that procrastination is an art best performed in great company. Thank
you.

\tableofcontents{}
\listoffigures{}
\listoftables{}
\lstlistoflistings
\printglossaries

\mainmatter{}

\documentclass[UKenglish,11pt,a4paper]{article}
\usepackage[utf8]{inputenc}
\usepackage[T1]{fontenc, url}
\usepackage{graphicx,babel,csquotes}
\usepackage[backend=biber,style=numeric-comp,sorting=nyt,maxbibnames=20,maxcitenames=20]{biblatex}
\addbibresource{macros.bib}
\addbibresource{references.bib}
\title{Ranklust - A bioinformatics solution to identify network biomarkers in cancer}
\author{Henning Lund-Hanssen}
\begin{document}
\maketitle
\tableofcontents{}
\part{Introduction}
To this day we still struggle with cancer. Even with all our modern equipment and knowledge we have still not been able
to tame this horrible disease. My thesis is about making a tool which gives cancer researchers an easier way of
identifying network biomarkers in cancer. 
\section{State of today}
\subsection{Biomarkers}
Biomarkers are at the centre of this thesis. They are what Ranklust should be able to detect and rank in order to
identify network biomarkers, and not just single molecules of them. A biomarker is a "biological measure of a biological
state" \cite{biomarker1}. It can be represented by the levels of a specific protein in our blood, a specific gene, or a
combination the two.

Biomarkers can be used for different purposes. They can be used to measure the effect of cancer drug treatment. That the
drug does what it is supposed to do. It can be used to predict disease development or the current stage of the
disease. Here is a list of what biomarkers currently can be used for:
\\\\
\textbf{Usages for biomarkers:} \cite{beyondpsa}
\begin{itemize}
    \item Disease disposition
        \begin{itemize}
            \item What is a patient's risk of developing cancer in the future?
        \end{itemize}
    \item Screening
        \begin{itemize}
            \item Does earlier detection of patients with cancer decrease morality?
        \end{itemize}
    \item Diagnostic
        \begin{itemize}
            \item Who has cancer? What is the grade of the cancer?
        \end{itemize}
    \item Prognostic
        \begin{itemize}
            \item What clinical outcome is most likely if therapy is not administered?
        \end{itemize}
    \item Predictive
        \begin{itemize}
            \item Which therapy is most appropriate?
        \end{itemize}
    \item Monitoring
        \begin{itemize}
            \item Was therapy effective? Did the patient's disease recur=
        \end{itemize}
    \item Pharmacogenomic
        \begin{itemize}
            \item What is the risk for adverse reaction to the prescribed therapeutic dose?
        \end{itemize}
\end{itemize}
\textbf{Characteristics for a good biomarker:}
\begin{itemize}
    \item Safe and easy to measure
    \item Cost efficient to follow up
    \item Modifiable with treatment
    \item Consistent across gender and ethnic groups
\end{itemize}
\subsection{Prostate specific antigen}
An example of a single molecule biomarker is the prostate specific antigen (PSA). This is a protein produced by the 
prostate gland in male humans. The identification of cancer with PSA is simple, the higher the level of PSA, measured in
ng/mL (nanograms per milliliter), the higher is the chance of the patient having prostate cancer \cite{cancerfacts}.

Today PSA is used for both identifying and evaluating the current stage of prostate cancer. This biomarker can be found 
by analyzing blood examples from patients, thus fulfilling some of the demands for a good biomarker,
but not all of them. It is easy to measure and easy to acquire, but not reliable enough to be used as the only marker
% find a better word, "tool" sucks!
to identify and determine the stage or remission of prostate cancer. The low grade of reliability comes from the fact
that even though higher levels of PSA shows higher chance of having prostate cancer, prostate cancer is not the only
reason to have elevated levels of PSA \cite{cancerfacts}.
% find other reasons? need to cite the cancerfacts twice?
Namely inflammation and enlargement of the prostate. Though, a man with both these cases may or may not develop prostate
cancer.

% Beyond PSA part


% PSA conclusion
% Part 1 of conclusion

% Part 2 of conclusion
So the conclusion for the PSA biomarker is that it is not reliable enough and are causing faulty treatment of prostate
cancer that may not even exist. Because even if a patient has prostate cancer, it may not be developing any further, 
promoting the case of not taking any action against it at all. So there is need for a new biomarker, or at least a 
better way of identifying them.

\subsection{Next-generation sequencing}
Today, rapid analyzing of genes and proteins are made available through Next-Generation Sequencing (NGS)\cite{ngs1}.
% Find a better source for this!
This opens up the possibility of looking at a bigger picture when trying to diagnose cancer patients. Through acquiring
more data, faster than before, there now exists databases with much information that is easy to access. This makes room
for building huge networks of proteins and genes, allowing for more extensively and thorough assays to be done. For
example, what if something that is classified as a prostate cancer biomarker only is viable when proteins that has not
been classified as a biomarker, also is present? Together they could represent a more appropriate
\emph{network biomarker}.
The amount of data that can be analyzed also opens up for another more personalized approach to each cancer patient.
Finding patient-specific biomarkers could make a huge impact on the quality of treatment \cite{personalized}.
\section{Next generation of biomarkers}
The PSA biomarker is over 20 years old \cite{psa-age}. Through those years, it has been discovered other and better
biomarkers for prostate cancer than PSA. Among those, PCA3, which is detectable through urine samples from patients. It
also has the benefit of not being affected by the size of the prostate gland \cite{pca3-size}. But still the results
could be better. Therefore, it has been tried to combine these two biomarkers in order to see if it is beneficial to
see the results from each biomarker in light of each other \cite{beyondpsa}. The results from these tests is that they
complement each other to a level of significance that makes it compelling to analyze them both to diagnose prostate
cancer. It is important to point out that even if these biomarkers are not the best at indicating if a patient as cancer
or not. Combined, these biomarkers are good at indicating progression and recurrence of prostate cancer.

But all of this is based on single genes or proteins. What if we looked at whole networks as biomarkers?
\section{Network}
\section{Clustering}
\section{Cytoscape}
% Rename the Cytoscape section to Software/Architecture/Development
% General description of development
Cytoscape is the open-source software platform the Ranklust app will be developed on. Its main purpose is to visualize
molecular interaction networks and biological pathways. It has an easy way to integrate \textbf{\textit{Apps}}, which may
be combined by other apps again to build big and complex applications which may solve problems in a bigger picture. The
goal of Ranklust is to cluster the networks we get from single gene prioritization and rank them in order to identify
network biomarkers. Apps taking care of making the networks already exists, but there still has to be made a decision
about whether or not they could be modified in order to better support the clustering.

% Databases
Which databases to use has to be considered. The reason to use databases is because they have information on how protein 
and genes form a network based on how they interact with eachother. The initial database candidates in Ranklust are
iRefIndex \cite{iri}, GeneMania \cite{gm} and STRING \cite{str}. These databases all have in common that it exists
Cytoscape apps made to use these databases. STRING however, does not have any repository available through the Cytoscape
app store, so interacting with the database through a new app in Cytoscape without making new plugins may be difficult.
On the other hand, both iRefIndex and GeneMania have their repositories easily available to the public together with
decent documentation. However, the difference between them is what they contain information about. iRefIndex contains
data about protein-protein interaction (PPI), while GeneMania contains data about genes.
% Check this next one out! Link from source!
Since proteins come from genes, GeneMania can also give us some information about proteins. Differences between the
two databases will be discussed in greater depth at a later stage.

The open-source plugins in Cytoscape to communicate with the databases are iRefScape \cite{iridb} for iRefIndex and 
GeneMANIA \cite{gmdb} for GeneMania.

% Cytoscape
Cytoscape is based on the Java programming language, which is a little bit untraditional for a software platform used to
develop bioinformatics tools. % cite needed!!!
The reason to choose Java above Python, Perl or other popular programming languages is simply because I am more
versed in Java programming than any other language. The Cytoscape community also promotes the idea of developing apps
that follow the \emph{OSGi} standard \cite{cytoscape-osgi}.

% OSGi and design patterns
Developing OSGi software should promote modularization % cite needed!!!
of the code and increase the probability of the app being launched as an official Cytoscape app, in addition to provide
other developers with the possibility of reusing my modules in their own apps. There exists several design patterns that
could prove to be useful in the development of Ranklust. Another strategy to follow may not be a direct design patter,
but more of a collection of them, is the clean code principles. More thorough examination of these strategies will
follow. % Cite needed for design pattern definition and clean code principles

% Prototyp section
Prototyping of different parts of the app will be done in Python and the Galaxy environment \cite{galaxy}. Python is an 
easy language to prototype with, and Galaxy is an easy environment to test small scripts with. Galaxy also has great 
logging of previous experiments combined with settings, so recreating simulations and comparing results is easy to do and
reliable. However, in my experience, Java comes up to par with Python in development speed once all of the boilerplate
code is written and a good deployment tool is used. Therefore the Cytoscape platform is used for the final deployment of
Ranklust.

% Maven section

\printbibliography
\end{document}

\part{Methods and implementation}
\label{pa:methods}
\chapter{Cytoscape}
\section{Intro}
The \textit{clusterMaker2} documentation for implementing new parts that is not
directly connected with clustering algorithms is non-existing. This part
explains the details about how it was done.

Cytoscape has a cookbook\cite{cytoscape-cookbook} listing a combination of best
practices and tips on how to start developing an app, add menues, panels,
algorithms, color schemes etc.. Instead of following all of these examples, we
have read through the existing code in clusterMaker2\cite{cm2-github} and tried
to structure the code of the contributions in Ranklust to match the structure
already implementet in clusterMaker2. Meaning that for each algorithm
implemented, they will all have a corresponding \textit{Factory} class and a
\textit{Context class}. For the panel that was implemented, it has a pure panel
class with Java Swing\cite{java-swing} settings, a panel task class, create
panel task class and destroy panel task class.

\section{Workflow and usage}
\subsection{Installation}
At the moment, clusterMaker2 does not officially contain Ranklust. So in order
to use Ranklust, the user are required to compile the Java source code or
download the updated JAR-file\cite{jar}(URL:
https://github.com/henninglh/ranklust-app/ranklust-1.0.0.jar). To build this
whole project I have used Maven as a build tool\cite{maven}. To compile the
project into an executable JAR-file Cytoscape can use, \gls{maven} has to be
instructed to skip all tests, because the current tests has compilation errors.

\textbf{Skipping tests with maven:}
\begin{Verbatim}[fontsize=\scriptsize]
mvn clean install -Dmaven.test.skip
\end{Verbatim}

The JAR-file that is being produced after the line over is executed in
a terminal or through the build tool of an \gls{ide}. This file has to be put in
the configuration folder of Cytoscape. The location of the configuration folder
can vary across operative systems and Cytoscape versions, but the Cytoscape
version 3.4.0 follows this path:
\begin{Verbatim}[fontsize=\scriptsize]
<user home folder>/CytoscapeConfiguration/3/apps/installed/<put the jar here>
\end{Verbatim}

After this step is done, the user can start up Cytoscape and should be able to
Ranklust.

\subsection{Researcher workflow}
The way researchers can use Ranklust is pretty simple. Step 1 and 2 will only be
required the first time the researcher want to use this app to solve his/her
problem. The rest has to be done every time, unless a previous session of
Cytoscape is loaded with more developed results.

\begin{enumerate}
    \item Install Cytoscape
    \item Install clusterMaker2 plugin through Cytoscape App manager
    \item Upload the network to be clustered and ranked into Cytoscape
    \item Use clusterMaker2 to cluster the network
    \item Use the Ranklust-part of clusterMaker2 to rank the clusters
        \begin{enumerate}
            \item Decide what to score the clusters on
        \end{enumerate}
    \item Use the Ranklust-part of clusterMaker2 to visualize the rankings
\end{enumerate}

Step 6 should show the researchers the ranking of clusters based on what
attributes they wanted to include in step 5a. This ranking represents cluster
biomarkers. The score of the clusters and the order they are ranked in will
decide the state of the patient. State of the patient can be many different
things and what the rankings will mean to the researcher is not yet final. It
will be a new indicator, a biomarker, and information about what it means has to
be gathered empirically through clinical research.

\section{Maven and the POM file}
The POM-file has to be updated because libraries connected to the pdf exporting
functionality is outdated. Updating the libraries, imports and the usage of
these libraries in the source code is enough to make the whole project compile.
Also, in order to get the third-party libraries to JUNG\cite{jung} into the
project, the three packages that is being used has to be added to the POM file
and the classes they contain has to be exposed in the OSGi
module\cite{osgi-felix}. Exposing the JUNG classes inside the clusterMaker2 OSGi
bundle could have been avoided if JUNG had OSGi modules in maven repositories,
but there is only 2 out of 3 OSGi ready modules that was needed in this project.
These were the changes needed:

Changed which packages was exported through the OSGi module.

\begin{lstlisting}[language=XML, caption={POM-file OSGi changes}]
<Export-Package>!${bundle.namespace}.*,*;-split-package:=merge-first</Export-Package>
\end{lstlisting}

Added these dependencies

\begin{lstlisting}[language=XML, caption={POM-file JUNG changes}]
<dependency>
    <groupId>net.sf.jung</groupId>
    <artifactId>jung-graph-impl</artifactId>
    <version>2.1</version>
</dependency>
<dependency>
    <groupId>net.sf.jung</groupId>
    <artifactId>jung-algorithms</artifactId>
    <version>2.1</version>
</dependency>
<dependency>
    <groupId>net.sf.jung</groupId>
    <artifactId>jung-api</artifactId>
    <version>2.1</version>
</dependency>
\end{lstlisting}

As seen here, the 3 modules needed is jung-api, jung-graph-impl and
jung-algorithms. Only the first 2 was OSGi ready. An alternative could have been
to create a OSGi ready module of the jung-algorithms module, but taking on the
responsibility for having a module updated at all times is too much for a single
person. However, it could become the clusterMaker2's developers responsbility to
create and update all 3 modules. The last alternative is to find other graph
ranking algorithm libraries like Apache Spark\cite{spark}, thas is OSGi ready.
The reason for not choosing Apache Spark is that it was not discovered until all
of the Java implementation was finished, making it a future goal to reach, at
best.

\section{App registration and java class connections}
First step of registration is to register a \textit{service listener} to the
already existing clusterMaker2 \textit{CyActivator}. Here I describe the name
of java methods in the \textit{clusterManager} class with strings. These two
methods for adding and removing ranking algorithms registers and unregisters the
algorithms to the \textit{App} menu in Cytoscape, making them accessible for the
user through the GUI. Registering a new service listener is a way of keeping the
Ranklust part out of clusterMaker2. Also, creating a standalone plugin at
a later stage will be easier if Ranklust is properly compartmentalized.

The \textit{RankFactory} class is a java public interface used for each of the
ranking cluster algorithms, \gls{hits}, \gls{pr}, \gls{prwp}, \gls{maa} and
\gls{mam}. A class applying the task factory design pattern is meant to deliver
an object of the class it is related to\cite{factory-design}, and it can return
different types of a class that has the same java
superclass\cite{java-superclass}. In this case, each class implementing the
\textit{RankFactory} will only have a single class to return. The RankFactory
class also creates a \textit{Context} class, which binds information the user
can input to the GUI to variables, allowing the algorithm to gain access to
parameter information about the algorithms before they run.

\section{Algorithm}
% Rank interface
\section{Ranking panel GUI}
The code for the \textit{RankingPanel} is a copy from the existing
\textit{ResultsPanel} in clusterMaker2. The results panel has some extra
information that is removed in the ranking panel. The ranking panel only
displays which clustering and ranking algorithm that was used and the network it
is used on, together with the score of each cluster sorted descendingly, having
the highest ranked cluster at the top. The ranking panel will display in the
same place as the results panel, to the right of the network view.

The panel supports multiple selection of clusters through the \textit{Control}
key on the keyboard, much like its existing behaviour on Windows when selecting
multiple directories or files in the \textit{File Explorer}. Selecting the
clusters in the panel will also select them in the network view, enabling the
user to use other Cytoscape utilities on the nodes/clusters selected, for
example create a new network based on only the selected nodes. The color of the
selected nodes will also change in the network view, when the user selects a
cluster in the ranking panel, providing visual feedback to the user, in order to
make it easier for the user to see where the selected nodes/clusters resides in
the network.

When tasking clusterMaker2 with showing the ranking panel, the color of the
nodes will also change. This is to visualize the rank of each cluster in the
network view to the user. Coloring the highest scoring clusters with red or
green and the lowest scoring with the opposite was the first and easiest
solution. There was made a choice to color the whole cluster according to the
rank it received, instead of individually color nodes according the their
contribution to the clusters rank. The user might be interested in choosing what
suits them best when it comes to this, so it might be implemented a menu for the
ranking panel in the future to best meet the users needs.

Taking color blind people into consideration red and green is not the
best combination. Colors resembling red and green in the form of hot-to-cold
colors that all types of colorblind people can view was therefore a criteria to
be met. A style satisfying these criteria will follow, having the hex value of
te color\cite{color-blindness3}, followed by the RGB value that was calculated
using a site that converts from hex to RGB values\cite{color-blindness2}.
\textit{\#ADD rgb\(170,221,221\)} for a blueish and cold color representing low
score. \textit{\#FEC rgb\(255,248,204\)} represents off-brown color that is a
step warmer than the blueish. The warmest color representing the highest scored
clusters is \textit{\#F99 rgb\(255,153,153\)}. The drawback with this color
combination is with people that has color blindness to the degree where they see
only greyscales. The colors low to high will go from grey to light grey to dark
gray, which does not seem logical. The color style might change if users
experience trouble having these colors representing the cluster ranks. 

\section{Current state of development}
The pull request\cite{git-pull-request} made from the Ranklust
contribution of this thesis has been accepted\cite{ranklust-accepted}, adding
3750 lines of code and deleting 1051. The changes was distributed across 50
files. The type of files edited ranges from the Maven POM-settings file and
.gitignore to plain Java source code files. Further changes will be made, as the
work on the Ranklust contribution will not end when this thesis is delivered.
One of the most recent changes to the contribution is that the
\textit{PageRankWithPriors} algorithm used to analyze the networks in this
thesis assumes undirected edges because of the
\textit{UndirectedSparseMultigraph} graph used. For normal use, directed edges
is believed to be the most sought after choice. Therefore, the pull request was
hardcoded with directed edges. Choosing between directed and undirected edges
should in the end be up to the users of clusterMaker2 and is on the TODO-list of
futures to implement.

\section{Known bugs}
\subsection{Menu bugs}
\paragraph{Unintentional execution of clustering algorithm}
\begin{enumerate}
    \item Open the clusterMaker2 clustering algorithm menu
    \item Select a clustering algorithm
    \item Exit the menu for the algorithm without running it
    \item Repeat step 1-2 for the same algorithm and it results in running the
        previous algorithm that was exited without displaying the parameter
        prompt
\end{enumerate}
This was the behaviour before Ranklust was added to clusterMaker2,
so it might be clusterMaker2 or it might be Cytoscape itself. Most likely, the
way clusterMaker2 algorithms is started through its \textit{TaskFactory}'s needs
to be changed, but this is just a hunch and not based on any analysis.

\subsection{Ranking panel bugs}
It is a known bug with the style for node coloring that the ranking panel
causes. The colors in the nodes might flicker when several ranking algorithms
has been run and the colors of each node has changed several times. This is
probably related to the previous color style for the nodes and it will be fixed
in the future.

\section{Clustering}
PPI networks already have a great deal of edges, and can be seen as clusters
that I should not alter. On the other hand, using clustering algorithms to make
clusters out of PPI-networks gives us the more control over the clusters and has
the potential to identify protein complexes\cite{ap-vs-mcl}. How big they should
be, how many of them I want, should I cluster on a certain attribute, or even
several? I have chosen to go with the \textit{Markov Cluster}(MCL)\cite{mcl}
clustering algorithm in \textit{clusterMaker2}, without any limitations to the
amount of clusters or weighting of the nodes. I prototyped Ranklust with
\textit{Affinity Propagation}(AP)\cite{affinity-propagation} because of its ease
of use and low execution time, but through a comparison between MCL and AP, it
was concluded with MCL having better performance in many aspects when it came to
cluster unweighted PPI networks\cite{ap-vs-mcl} with binary interactions. These
aspects being noise tolerance and more robust behaviour, which will contribute
greatly to cluster the iRefWeb network.

\section{Ranking clusters}
I have used five algorithms to rank the clusters, Multiple Attribute Additive
Method (\gls{maa}), Multiple Attribute Multiplication Method (\gls{mam}),
PageRank (\gls{pr}), PageRank with Priors (\gls{prwp} and Hyperlink-Induced
Topic Search (\gls{hits}). The first two are simple algorithms that through
addition or multiplication calculates a cluster's average score. The three
others utilize network ranking algorithms from the Java JUNG\cite{jung} library.

Every algorithm implemented in Ranklust for ranking clusters except \gls{hits}
takes node and/or edge scores as input for calculating the scores for each
cluster.

\subsection{Multiple Attribute Additive Method (MAA)}
Multiple attribute additive method is the first algorithm implemented.  The user
has the option of choosing an unlimited amount of attributes from nodes and
edges.

It goes through all of the nodes in each cluster and sums up the number-
attributes the user chose. Each cluster is then ranked based on the average sum
in each cluster and ranked descending, with the highest ranking cluster as the
most likely prostate cancer biomarker cluster.

There is a question as to how to rank the edges in the cluster. We chose to rank
each edge as it is listed to the user in Cytoscape. So if it is listed only once
time in the edge table, it will only be scored additively once. This decision
was based on simplicity. To not represent the edge as something the user did not
define it as, or is unable to understand. Some clustering algorithms might
assign the same node or edge to several clusters, though this is not the case
with the algorithms I use in this thesis. Support for this is only implemented
by \gls{maa} and \gls{mam}, as they were implemented before a final decision on
which clustering algorithms should be used. If \gls{maa}/\gls{mam} discovers
this special case of several scores for a single node or edge, it will assign it
the highest value.  The reason for leaving this feature in Ranklust is to have
an example on how it can be done, should it be a problem in the future.

\subsection{Multiple Attribute Multiplication Method (MAM)} Multiple attribute
multiplication method is to some degree redundant, considering what exists from
before in Ranklust. The only difference from \gls{maa} is the scale the scores
will be in. \gls{maa} adds the scores from each node and edge in the cluster
through addition, \gls{mam} does it with multiplication.

A problem that occurs with multiplication is calculating scores for clusters
that contain nodes with a score between 0 and 1, since the score would decrease
to such a degree that it would be difficult to work with when normalizing the
scores. The solution I have chosen for this problem is to make a new score from
the score that is to be added to the cluster average, and add 1.0 to it. This
way, when the existing average score in the cluster is multiplied with the new
score, it will always increase, unless the old value was 0.0, or both the old
and the new value was 1.0. In the case of values above 1, they will also be
given an increase by 1.0 in order to keep it consistent if the scores vary
between 0 to \textit{n}. Normalizing every value before running the algorithm
contributes to keep all of the values between 0 and 1, and in that way prevent
scaling problems when adding 1.0 to a score over 1.0. The whole reason to add
1.0 was to counteract the problem with the score decreasing when it should be
increasing.

\subsection{PageRank (PR) and PageRank with Priors (PRWP)} 
A Random Walks\cite{random-walks2} algorithm which used priors, called
seed-weighted random walks ranking \cite{sw-rwr}, proved to be effective at
prioritizing biomarker candidates. PageRank (\gls{pr}) is an algorithm based on
the Random Walks principle, and it is contained inside the Java library
\textit{JUNG}\cite{jung}. PageRank was previously used by Google to rank
webpages \cite{pagerank}. PageRank with Priors (\gls{prwp})\cite{pr-bio} is a
modified version of \gls{pr}, where nodes and edges can be assigned a score
prior to the \gls{pr}'s traversal of the network. \gls{pr} can have values
assigned to the edges, but it does not require any scores in order to rank
clusters, which \gls{prwp} does. PageRank with Priors is abbreviated \gls{prwp}
because of the Java classname it has in the Java JUNG library
- \textit{PageRankWithPriors}.

The difference between \gls{maa} and \gls{mam} compared to \gls{pr} and
\gls{prwp} is how the network is scored. \gls{maa} and \gls{mam} calculates the
score for each cluster by summing up the attributes in edges and nodes according
to the cluster attribute. \gls{prwp} scores the current network regardless of
the clustering attribute. 

MCL gives the option of creating a clustered network, which opens up the
possibility of working with two types of the same network, non-clustered and
clustered. They both have the clustering attribute in the network, edge and node
table, so that the ranking algorithms are able to score the clusters. \gls{prwp}
scores the currently selected network in Cytoscape, resulting in the option of
scoring the non-clustered network or clustered network. The last option gives
the clustering algorithm a bigger impact on the score, because the clustered
network has perturbated edges between nodes that is not in the same cluster. MCL
in Cytoscape can show the "inter-cluster" connecting edges, which is the egdes
that was perturbed during clustering. This last option is a combination of the
two others. It will be visually close to the clustered network, but algorithms
that run on the network with inter-cluster edges will have the same result as
the network only containing the cluster attribute.

Implementation wise, there is also a difference in how the scores are stored in
the network after the algorithm is finished executing. All the scores are stored
in the nodes. To give information to the user about the scores the edges
received, each edge will display the total score for the cluster it is a member
of, just like the nodes. Only \gls{prwp} and \gls{pr} will also display the
single node score.

\subsection{Hyperlink-Induced Topic Search (HITS)}
Hyperlink-Induced Topic Search, \gls{hits}, an algorithm that is similar to
PageRank, was developed around the same time \cite{hits}\cite{hits-origin} and
is also contained within the Java JUNG library. \gls{hits} will not be used to
calculate cluster scores and ranks, due to the fact that it does not require any
form of weighting in nodes or edges. Though, it can be used in combination with
other ranking algorithms through the use of \gls{mam} or \gls{maa}. 

An example of this could be running \gls{prwp} with a score attribute, then
\gls{hits} on the same network. The next step would be to combine the two scores
from \gls{prwp} and \gls{hits} with \gls{maa}. Though, the idea of achieving
novel information on network biomarkers for prostate cancer through this
workflow is pure speculation, and we have chosen to use each ranking algorithm
separately, in order to limit the amount of datasets to analyze.

\subsection{PR, PRWP and HITS}
Because \gls{pr}, \gls{prwp} and \gls{hits} rely on an alpha variable
controlling the probability of a reset in the traversal of the network. This
"traversal reset" is automatically triggered if the algorithm hits a node with
no outgoing edges. As the clustering algorithms might change the edges in the
network, the outcome of performing ranking on a cluster-created network, versus
a network with only a cluster attribute to identify the clusters, can
potentially be vastly different. I have decided to not use the cluster-created
network, and instead use the network with cluster attributes. The perturbation
of edges in the network and separation of nodes into a more disconnected graph
might hide interactions in the network that can provide useful information to
\gls{prwp}. On the other hand, it might produce more noice in the network and
skew the ranking of the clusters through interactions that are false positives,
regarding protein complex interactions. 

The alpha value parameter will be set to 0.3. This specific alpha value has been
proven to be effective in both identifying and prioritizing candidate genes in
diseases\cite{disease-prwp}. Another parameter for \gls{pr}, \gls{prwp} and
\gls{hits} is the iteration parameter. For each iteration of these algorithms,
they assign a value to each node. Performing multiple iterations contribute to
stabilize these values. In the previously mentioned report from Google
\cite{pr-parameters}, a \textit{n x n} matrix where \textit{n} was about 25
billion, required about 50 - 100 iterations in order to stabilize the node
values. Iterations needed to stabilize increase with the size of the network,
therefore the iteration value to start with will be 30. It is a little less that
what Google has specified, but in contrast, the iRefWeb network is 9500 nodes
big, with approximately 43,000 edges, which is a considerably smaller network
than "the internet", which Google tried to rank.

\chapter{Programming specifics}
\section{Tables vs pure OO}
One thing in particular that deserves to be mentioned is the way networks are
handled. The \textit{CyNode} objects, which represents a single node, does not
contain much. This is because all of the information is saved in the form of
plain cells in a spreadsheet. This may at first seem like a way to make it easy
to show information to the user, but it is also a way of working more efficient
with network graph data. Creating objects with attributes for each node in a
huge network will increase the amount of overhead. Working with all of the
information in the way of a spreadsheet with rows and columns results in a
decrease in overhead. A new node is a new row in the node table, so in relation
to building the network structure, it is not a complex abstraction.  The
difference comes in when the nodes have several attributes. 

In a table or spreadsheet, attributes can be represented as a single column and
be created once for the whole network, instead of once for each node object
created. This assumes that getting the objects out of the table is possible by
either indexing on a number for arrays, or a unique key for map structures.
The result is both a lookup, insertion and deletion time of O(1). These times is
as low as the Big-O notation goes in terms of speed related to the size of
a collection inside a data structure. So both the creation of objects goes
down, and the retrieval of attribute information is as low as it is possible to
get, when I choose to represent the time by Big-O notation. The only drawback
with this implementation is that there is no current type of wrapper around the
row and column system. So the retrieval of information is not done the most
intuitive way. This is the way Cytoscape works as a whole, so changing the
way this works can either be done through changing the Cytoscape source code, or
implementing a wrapper as a standalone function inside clusterMaker2 or as
a standalone plugin. In Ranklust, an extension to the existing
\textit{NodeCluster} class was used to keep scoring information about nodes in
clusters.

\section{Changes in existing classes}
\subsection{NodeCluster}
\begin{Verbatim}[fontsize=\scriptsize]
/**
 * In it's simplist form, a Cluster is a group of nodes that represents the
 * nodes that are grouped together as the result of a clustering algorithm
 * of some sort.  A more complicated form of a cluster could include clusters
 * as part of the list, which complicates this class a little....
 */
\end{Verbatim}

This comment is in the NodeCluster class. In the Ranklust implementation,
several attributes and methods has been added to this class. Saving the
temporary state of the scores to the clusters could have been done in the node
and edge-tables in Cytoscape, but it was faster both implementation- and
performance-wise to extend the NodeCluster class. Also, a criteria for extending
this class was that the existing API of this class should not change, in order
to avoid unneccessary changes for other classes in clusterMaker2, that is not
a part of Ranklust. 

Additions to NodeCluster contains variables representing cluster \textit{score},
\textit{rank} and a \textit{HashMap} from a cluster-node's SUID to its score.
The map and rank variables is part of a previous implementation and can be
removed. Adding node scores to the cluster and normalizing it afterwards.

\textbf{Common responsibilities introduced in Ranklust}
\begin{itemize}
    \item Adding node scores to the cluster (ranking algorithm related)
    \item Normalizing cluster scores for a list of clusters (static method)
    \item Calculating min/max/avg scores for a list of clusters (static method)
    \item Ranking a list of clusters by their score and set their rank
        accordingly (ranking panel related)
\end{itemize}

\subsection{GraphicsExportPanel}
Minor changes were done to GraphicsExportPanel when a new class for handling PDF
documents was introduced. This was the result from the need to change the maven
dependency to use the "com.itext"\cite{itext} repository location instead of
a location from "com.lowagie" because the library was relocated
\cite{lowagie-to-itext}.

\section{New classes}
\subsection{ClusterUtils}
\textit{ClusterUtils} is a new class implemented in the Ranklust contribution.
It is used by every cluster ranking algorithm implemented in Ranklust. To some
degree it has a common responsibility with the \textit{ModelUtils} class from
clusterMaker2. The difference being that ClusterUtils is focused on inserting
scores in tables related to cluster ranks. It also normalizes scores not
directly related to a NodeCluster, but rather a group of \textit{CyNodes}, that
has mapped their SUID in Cytoscape to a score received from a ranking algorithm.

\textbf{Common responsibilities introduced in Ranklust}

\begin{itemize}
    \item Getting cluster attribute based on CyNetwork
    \item Getting ranking attribute based on CyNetwork
    \item Ranking clusters firstly by score, secondly by cluster number
        (assuming a small cluster number is a bigger cluster)
    \item Fetching ranking results based on CyNetwork (ranking panel utility)
    \item Fetching clusters based on CyNetwork (ranking algorithm utility)
    \item Add score to a NodeCluster based on specified attribute column and
        CyRow to get it from
    \item Insert cluster scores into the default node- and edge-tables
    \item A simplified way of creating columns based on a specific column name,
        attribute class, table to create column in and immutability status
\end{itemize}

\section{Handling nodes and edges}
Recipe for working the nodes and edges: The steps are almost equal for
calculating both node and edge scores. How the edges are handled depends on what
direction they have. Are they undirected, directed, or unidirected. In the
current version of Ranklust, \textit{PageRankWithPriors} assumes undirected,
\textit{PageRank} and \textit{\gls{hits}} assumes directed. \textit{\gls{maa}}
adds the score from the edge to the cluster and \textit{\gls{mam}} multiplies
the highest score found as an edge attribute with the current cluster score.
\gls{mam} and \gls{maa} assumes the score is directed corresponding with source
and target nodes in the edge table of Cytoscape. If an edge is a part of
several clusters, each cluster will add the score of the edge to its average
cluster score. For PR and PRWP, the score will always reside within nodes and
not edges at the end of execution. To distribute the scores among the edges, a
traversal through all of the edges, which is being matched against all of the
nodes in each cluster is performed. When the traversal for an edge hits a
cluster with a node that matches either the target or source node in the edge,
the cluster that has this matched node adds its cluster rank score to the edge.

\begin{enumerate}
    \item Add score to the cluster
    \item Repeat step 1-2 for every edge
    \item Sort clusters based on rank and create a column to represent the
        cluster score
    \item Create single node score from cluster
\end{enumerate}

\chapter{Data pre- and post-work}
Interpreting these results was done solely by Python scripting, from formatting
the data retrieved from databases in order to construct networks in Cytoscape,
to cleaning the results exported from Ranklust, and analyzing the cleaned data. 

\section{Before using Ranklust}
\subsection{Network creation}
The network was created with protein interaction data from iRefWeb
\cite{irefweb}. The query was constructed with the idea of having as large
a network possible of human-only interacting proteins, that at a later stage
could be converted into genes. The tabs that is not expanded was not changed
from the standard settings when creating such a query, which starts off with all
of the checkboxes unchecked (\ref{fig:irefweb}). The picture of this query
states that there was 109003 iterations. However, this number is subject to
change several minutes after querying iRefWeb. This picture of the query was
taken before the database search for interaction had finished, which is the
reason for displaying 273 interactions less than the actual amount used in the
network that Cytoscape ran the algorithms on.
\begin{figure}
    \centering
    \includegraphics[width=15cm]{mitab_lite_109276}
    \caption{iRefWeb network query}
    \label{fig:irefweb}
\end{figure}

\subsubsection{iRefWeb}
It was downloaded in the MITAB-MINI format. A python script cleaned up the file,
leaving only the aliases for the source and target proteins. This result was
matched up against genes from HGNC. The HGNC data represents genes, and their
protein product. This way, the network would consist of genes instead of
proteins. This conversion was done in order to compare the end results from
ranking the clusters with genes that has connections to prostate cancer.
A criteria for the interactions retrieved from iRefWeb when converted from
proteins to genes was that both the source and target protein had to have
a match in the HGNC data, to not create any combinations of gene to protein or
protein to gene interactions. Duplicate interactions was not taken care of and
they did not occur in the network file either.

\subsubsection{COSMIC}
A network with only prostate cancer related genes from
COSMIC\cite{cosmic-download} was constructed with interaction information from
the iRefWeb network. This network was much smaller in terms of both nodes and
edges (interactions). The COSMIC network was created and ranked to see how
ranking a network with only cancer relevant genes would result in when data from
jensen was used to compare it to a much larger and generalized network.
Filtering the genes from COSMIC was done the same way as filtering the scores
for the genes; using only the COSMIC genes that contained genes from both ends,
source and target gene, of an interaction in the network.

\subsection{Score creation}
The golden standard for scoring that I used was created by data from
DisGeNET\cite{disgenet} and \gls{dragon}\cite{dragon}. 

DisGeNET is a database that is one of todays largest repositories of information
between diseases and genes.  It integrates expert-curated data with text mined
data. DisGeNET provides a score for genes related to a specific disease, in this
case prostate cancer.  This score is based on "the supporting evidence to
prioritize gene-disease associations"\cite{disgenet}. It also has a Cytoscape
plugin, but like iRefScape, it is outdated and can not be used together with
Cytoscape version 3.4.0. DisGeNET was chosen because of its comprehensive
collection of prostate cancer gene scores that was used as prior scores in both
\gls{prwp} and \gls{maa}.

\gls{dragon} is a knowledgebase of genes that is been experimentally verified as
an implicator in prostate cancer. It offers a bigger variety of information on
prostate cancer in general, as opposed to DisGeNET, but it does not have scores.
Therefore, to be used in the \gls{golden}, the genes from \gls{dragon} had to
receive a constructed score comparable to the scores from DisGeNET.

The DisGeNET data came with decimal scores from 0 to 1 and contained data about
several diseases, not only prostate cancer, and multiple instances of the same
gene could have multiple scores for different types of prostate cancer. The
duplication of gene entries with scores related to prostate cancer resulted in
a solution where the average of all of the prostate cancer scores created
a single score for a single gene. However, the data from \gls{dragon} did not
have any scores and was just a list with genes relevant to prostate cancer.  To
combine the two lists, the \gls{dragon} scores had to be converted to match the
DisGeNET scores. The method for converting the scores was to let the DisGeNET
scores be left as they were and try to add scores to the genes from the
\gls{dragon} data list. An average was created from the maximum and minimum
value from the DisGeNET scores. Then some sort of "sane" max value was created
by adding the average to a quarter of the value of the average value subtracted
from the maximum value.  This created a final value between the average and
maximum value that all of the \gls{dragon} genes received. The \gls{dragon}
genes receives on average a higher score than the DisGeNET genes because of the
way the lists are constructed. \gls{dragon} has no score, therefore they
signalize a proof of relevance to prostate cancer, while DisGeNET has a score
for how relevant the gene may be.

Finally, this golden standard of scores was filtered by removing any genes that
did not occur in the network. Because of the cross-validation performed after
the ranking of the clusters, the scores could not contain any genes not in the
network in order to not remove genes from the files with scores that would not
occur in the network. This would have resulted in a cross-validation where
a 100\% match never could happen.

\subsection{Creating scores for cross-validation}
A form of cross-validation was performed to assess how the ranking algorithms
performed alone and compared to each other. Creating the cross-validation scored
data sets required information about the clusters. Running MCL clustering on the
network and then export the node table in Cytoscape created the cluster basis
needed. From these clusters, it was possible to identify which clusters
contained scores that could be removed during cross-validation. A random 10\% of
the genes used for scoring was removed. Though, the amount of genes removed was
not totally random. There was set a rule that every cluster that contained genes
with a score from the golden standard, had to end up with keeping atleast 1 of
those genes. So in a cluster of 4, where 2 of them were scored in the golden
standard, that cluster could only lose 1 of those scored genes when the 10\% was
removed. Had this rule not been applied to the cross-validation, there would not
be a guarantee that 100\% of the removed genes would be found as candidate
genes. For \gls{maa} ranking, it would be impossible, and for \gls{pr}WP it
would be possible, but much less likely to find them. This is due to the simple
fact that clusters that go from having weighted nodes (genes), to having no
weighted genes, should not be ranked very high. The point of having weights is
to rely on prior information about cancer relation in genes and find cancer
candidate gene clusters by applying reason for "guilt-by-association" to the
genes that are not weighted.

\section{After running Ranklust}
\subsection{Exporting the data}
The data from performing Ranklust on the networks was exported as node tables
through Cytoscape. These tables contains information about each node, but not
any information about the edges. The scores added to the networks did not
involve any edge information, and for both of the algorithms used, \gls{maa} and
\gls{prwp}, the calculated scores are stored in the cluster, so there was no use
at this point to use the information in the edge table to find the cancer
candidate genes.

\subsection{Cleaning the data}
Cleaning the data was done in Python with its table manipulation library,
Pandas\cite{pandas}. The cleaning removed unnecessary data from the node table
files from exported from Cytoscape. The data still represents the same before
and after the cleaning, the difference is that the cleaned version is has tab
separated data, for better visual interpretation and it has less columns.

\textbf{Before cleaning:}
\begin{Verbatim}[fontsize=\scriptsize]
"SUID","__mclCluster","name","PRWP","PRWP_single","score","selected","shared name"
"72","25","FAM160B1","0.04478373407949736","7.818032875772688E-4",,"false","FAM160B1"
"73","6","APOPT1","0.04119274479140687","9.848244844960219E-4",,"false","APOPT1"
"74","6","FAM84A","0.04119274479140687","9.848244844960219E-4",,"false","FAM84A"
\end{Verbatim}

\textbf{After cleaning:}
\begin{Verbatim}[fontsize=\scriptsize]
__mclCluster	name	PRWP	score	PRWP_single
1136.0	CDC25C	1.0	0.272714418721	0.225926983193
1136.0	LZTS1	1.0	0.122995792115	0.197642927074
690.0	CREB3L4	0.918902810704	0.272714418721	0.189744545774
690.0	SCX	0.918902810704	0.0	0.102170140032
\end{Verbatim}

\begin{itemize}
    \item \_\_mclCluster
        \begin{itemize}
            \item Which cluster the gene/node in this row belongs to
        \end{itemize}
    \item name
        \begin{itemize}
            \item The HGNC representation of this gene/node
        \end{itemize}
    \item PRWP 
        \begin{itemize}
            \item The score of the cluster to this gene/node belongs to
        \end{itemize}
    \item score 
        \begin{itemize}
            \item The prior score assigned to this gene/node
        \end{itemize}
    \item PRWP\_single
        \begin{itemize}
            \item The PageRank score assigned to this gene/node (not a column in \gls{maa})
        \end{itemize}
\end{itemize}

\subsection{Recreate the clusters}
The cleaned data was then aggregated into clusters represented in text format.

\textbf{Combined clusters:}
\begin{Verbatim}[fontsize=\scriptsize]
clusters	score	genes
1136	1.0	LZTS1:0.122995792115,CDC25C:0.272714418721
690	0.918902810704	SOX9:0.272714418721,SCX:0.0,CREB3L4:0.272714418721
1004	0.758524309337	MMP14:0.272714418721,MMP13:0.12
\end{Verbatim}

Here, the \textit{clusters}-column are the unique identifier, and not the
\textit{HGNC} identifier or the \textit{SUID}from the previous data. The score
column does not represent the same data as before, but rather the same as the
PRWP column displayed in the cleaning examples, the total score of the cluster.
The genes column is double separated. The column itself is tab separated from
the other columns, the genes in it is separated with commas and the prior scores
belonging to each gene is delimited by a colon, not the \gls{prwp}/\gls{maa} or
any other algorithmic related score. This column contains all of the genes in
the cluster and aids in distinguishing the already proven biomarkers for cancer,
and the candidates.

\subsection{Cross-validation comparison}
Both \gls{prwp} and \gls{maa} had 10 cross-validated results each. Each
individual result from both of the algorithms was compared with the \gls{golden}
result created by each of the algorithms. This procedure was done to identify
the cancer biomarkers that existed in the \gls{golden} but ended up being
candidate cancer biomarkers in the cross-validation.

\subsection{Comparing with test data}
Matching the cluster ranks against test data is the last step in terms of
creating results that could reveal candidate network biomarkers for prostate
cancer. The procedure for matching cluster ranks with test data will consist of
inserting the test data genes with the cluster ranks. The test data genes will
be a single column list which contains unique genes proven to be prostate cancer
biomarkers. Creating a table with an overview of the test genes that match
the genes inside the ranked clusters, and whether the \gls{golden} classified
these ranked genes as prostate cancer genes or prostate cancer candidate genes,
is the final goal of this thesis. Traversing the ranked clusters and ascertain
which genes from the test data is inside which clusters and if they are
candidate biomarkers or already existing ones will end up producing the final
table used to represent the core result in this thesis. From creating and
scoring a network, cluster it, ranking it, assert its suitability to rank
clusters and present a final list of network candidate biomarkers for prostate
cancer.

The test data will come from three different sources. Manually expert curated
genes from the movember group, prostate cancer genes from experiments recorded
in COSMIC\cite{cosmic-download}, and proven lethal prostate cancer genes from an
experiment that tried to reduce the amount of over-treatment of prostate cancer
based on screening from PSA\cite{psa-overtreatment}.

\section{Plot creation}
The plots was created with an online tool named Plotly\cite{plotly}. Throughout
the whole process of creating plots, the X-axis represented the cluster ranks,
ordered from best to worst scored cluster, left to right, with ascending values
representing the ranks. The Y-axis represented in all cases an average. This
average was always calculated by dividing the sum of a specific value for each
gene, by the total number of genes in the cluster.

\part{Results}
\label{pa:results}
\chapter{Graph analysis}
PPI network from irefweb.
downloaded with these settings: %picture from irefweb
irefweb said 27887 connections % Correct this!
only uniprot and refseq
uniprot said
\section{Creating a network}
\subsection{Creating connections}
\subsection{Adding weights}
\section{Ranking results}
Talking about MCL results (Table \ref{tab:mcl-inflation})
\begin{table}
    \centering
    \begin{tabular}{| l | c | c | c | c | c | c |}
        \\textbf{Inflation} & \textbf{Number of clusters} & \textbf{Average
    cluster size} & \textbf{Maximum cluster size} & \textbf{Minimum cluster
    size} & \textbf{Modularity} & \textbf{Edges}\\
        \hline
        1.6 & INSERT & INSERT & INSERT & INSERT & INSERT \\
        1.8 & 1400 & 6.60 & 660 & 2 & 0.307 & INSERT \\
        2.0 & 1599 & 5.68 & 405 & 2 & 0.269 & INSERT \\
        2.5 & 2053 & 4.20 & 179 & 2 & 0.223 & INSERT \\
        3.0 & 2210 & 3.75 & 122 & 2 & 0.199 & 6744 \\
        \hline
    \end{tabular}
    \caption{MCL clustering parameter and statistic results}
    \label{tab:mcl-inflation}
\end{table}
\section{Parameter testing}
\section{Cross validation}
\begin{figure}
    \label{fig:irefweb-prwp}
    \includegraphics[width=15cm]{Distribution of iRefWeb network (PRWP)}
\end{figure}
\begin{figure}
    \label{fig:irefweb-maa}
    \includegraphics[width=15cm]{Distribution of iRefWeb network (MAA)}
\end{figure}
\begin{figure}
    \label{fig:cosmic-prwp}
    \includegraphics[width=15cm]{Distribution of COSMIC cancer network (PRWP)}
\end{figure}
\begin{figure}
    \label{fig:cosmic-maa}
    \includegraphics[width=15cm]{Distribution of COSMIC cancer network (MAA)}
\end{figure}

\section{Jensen comparison}
\begin{figure}
    \label{fig:txt-irefweb-prwp}
    \includegraphics[width=15cm]{Distribution of textmined genes in iRefWeb network (PRWP)}
\end{figure}
\begin{figure}
    \label{fig:txt-irefweb-maa}
    \includegraphics[width=15cm]{Distribution of textmined genes in iRefWeb network (MAA)}
\end{figure}
\begin{figure}
    \label{fig:know-irefweb-prwp}
    \includegraphics[width=15cm]{Distribution of knowledge mined genes in iRefWeb network (PRWP)}
\end{figure}
\begin{figure}
    \label{fig:know-irefweb-maa}
    \includegraphics[width=15cm]{Distribution of knowledge mined genes in iRefWeb network (MAA)}
\end{figure}
\begin{figure}
    \label{fig:exp-irefweb-prwp}
    \includegraphics[width=15cm]{Distribution of experimental genes in iRefWeb network (PRWP)}
\end{figure}
\begin{figure}
    \label{fig:exp-irefweb-maa}
    \includegraphics[width=15cm]{Distribution of experimental genes in iRefWeb network (MAA)}
\end{figure}
\begin{figure}
    \label{fig:txt-cosmic-prwp}
    \includegraphics[width=15cm]{Distribution of textmined genes in COSMIC cancer network (PRWP)}
\end{figure}
\begin{figure}
    \label{fig:txt-cosmic-maa}
    \includegraphics[width=15cm]{Distribution of textmined genes in COSMIC cancer network (MAA)}
\end{figure}
\begin{figure}
    \label{fig:know-cosmic-prwp}
    \includegraphics[width=15cm]{Distribution of knowledge mined genes in COSMIC cancer network (PRWP)}
\end{figure}
\begin{figure}
    \label{fig:know-cosmic-maa}
    \includegraphics[width=15cm]{Distribution of knowledge mined genes in COSMIC cancer network (MAA)}
\end{figure}
\begin{figure}
    \label{fig:exp-cosmic-prwp}
    \includegraphics[width=15cm]{Distribution of experimental genes in COSMIC cancer network (PRWP)}
\end{figure}
\begin{figure}
    \label{fig:exp-cosmic-maa}
    \includegraphics[width=15cm]{Distribution of experimental genes in COSMIC cancer network (MAA)}
\end{figure}

\section{Comparison to known biomarkers}
\section{Identification of possible biomarkers}

\part{Conclusion}
\label{pa:conclusion}
\chapter{Future work}
\section{General improvements}
We have used an undirected network, which is not the preferred type to use with
PageRank, but it works. An idea could be to use KEGG pathways\cite{kegg}, which
has the option of downloading directed networks of several types.
STRING\cite{str} also has some directed network information, together with
weights. Because of STRINGs size, utilizing a Neo4J database to perform the
ranking algorithms could be a better option than directly in Cytoscape.

In the future, maybe a database with complete protein complexes could exclude
the need for clustering, and provide different ways of ranking them based on
query criterias.

Other ranking algorithms than PageRank could also be used. There are numerous
variations based off of this well known algorithm, for example NetRank and
GeneRank\cite{netrank,generank}. Not all of these PageRank variants are intended
to be used with PPI networks, but they can in many cases be modified to fit this
specific purpose.

\section{Minor features to complement Ranklust/clusterMaker2}
A compiled list of all considered features and tweaks in the Ranklust
contribution to clusterMaker2

\begin{itemize}
    \item User can specify what direction the ranking algorithm should consider
        for the graph
\end{itemize}

\section{Ranking through Neo4J}
ClusterMaker2 does every calculation within Cytoscape. There exists another way
of doing large and complex calculations, especially when it comes to algorithms
that focus on edge information in the network. CyNeo4j\cite{cyneo4j} is
a Cytoscape App that can realise this idea. It supports import/export of data
with a Neo4J\cite{neo4j} database. The original CyNeo4J Github-repository does
not support user authentication at the moment, but during the discussion about
what ranking algorithm should be used, Neo4J was an alternative. It resulted in
a git fork\cite{git-fork} of the CyNeo4J repository and simple user/password
authentication was added.

\subsection{Data communication}
Data communication between the Neo4J database server and the Cytoscape instance
is done through the CyNeo4J app to Cytoscape \cite{cyneo4j}. CyNeo4J is a
Cytoscape app that was developed during the Google Summer Code 2014 arrangement.
It supports connecting to a Neo4J instance, as well as syncing data up and down
from and to the database server. One thing it did not support was authentication
on Neo4J servers. Not having authentication is a serious problem, so we
implemented a simple way of getting access to the database server by providing a
possibility to insert username and password at the same time the user has to
provide a URL to the Neo4J database server instance. Implementation-wise, this
only required an extra header to be included in each http request going to the
password protected Neo4J database server instance. Every request used the static
\textbf{Request} class to Get/Post/Put HTTP requests to the Neo4J database
server instance. Except for creating the Auth64 encoded information, the
refactor looked something along these lines in all of the files.

Before:
\begin{lstlisting}[frame=single,language=Java]
Request req = Request.Post(url)
        .bodyString(call.getPayload(), ContentType.APPLICATION_JSON);
\end{lstlisting}

After:
\begin{lstlisting}[frame=single,language=Java]
Request req = Request.Post(url)
        .addHeader("Authorization", auth64EncodedInfo)
        .bodyString(call.getPayload(), ContentType.APPLICATION_JSON);
\end{lstlisting}

Synchronization time between Neo4J and Cytoscape through the CyNeo4J app is a
huge timesink. As of now, the time it takes to populate an empty Cytoscape
network with the gene information from STRING is about 2 hours, though on a slow
laptop. This could be shortened by exporting the Neo4J data with GraphML and
into Cytoscape. Because after the initial data is inside Cytoscape, updates to
the Neo4J instance goes much faster.

The CyNeo4J also uses a legacy HTTP library to get information from the Neo4J
database \cite{legacy-neo4j}. It is possible that the performance increases with
the new library \cite{transactional-neo4j}. The new library supports creating
transactions, which implicit gives support for rollbacks in case something goes
wrong with the query.

A future improvement to the CyNeo4J app could be to change the communication
between Neo4J and Cytoscape to be done in GraphML and not Cypher. This is
because through this whole process of importing and exporting data from and to
Neo4J, GraphML has shown itself to be a superior format over Cypher. Though, to
this day, direct queries to a running Neo4J instance has to be done in Cypher,
and is not possible in GraphML.

\backmatter{}
\printbibliography
\end{document}
